\documentclass[12pt]{article}
\usepackage{amsmath}
\usepackage{amssymb}
\usepackage{amsfonts}
\usepackage{diagrams}
% \usepackage{timing}  % debian: texlive-latex-extra

\begin{document}
\title{{\LARGE \textbf{The Sheep}}}
\author{\large TOM SCHOUTEN}
\bibliographystyle{acm} \maketitle

\begin{abstract}
  This document describes The Sheep, a simple 1--bit sound synth
  implemented in Staapl's PIC18 Forth dialect \verb|staapl/pic18|, and
  can serve as a tutorial.
\end{abstract}


\section{Introduction}

This document presents Sheep, an application written in Staapl's PIC18
Forth dialect, henceforth called ``the Forth''. It serves as an
introduction to language oriented programming through writing a
\emph{domain specific language} (DSL).  The DSL will deal mostly with
the construction of abstractions for control flow through the use of
procedures and macros.

The intended audience consists of the two fields I'd like to bring
together with Staapl: those familiar with a higher level language like
Scheme and those familiar with Forth or low level assembly
programming.  If for some reason the subject of this document
interests you but it is too dense, please comment on the Staapl
mailing list.

\section{Building a Language}

% The main problem in programming is \emph{separation of concerns}. If
% I'm writing 

The main idea behind \emph{bottom up} program design is to gradually
increase the level of \emph{abstraction} by constructing a high level
language in terms of a low level language.

Most languages support some form of \emph{procedures} as their basic
means of abstraction. This gives a single name for a bunch of
operations. The names of these procedures form an \emph{interface} to
their functionality.

In language like Forth (or Lisp) which have simple syntax,
\emph{macros} are available as an additional abstraction mechanism.
They allow the construction of abstractions that cannot be expressed
as procedures alone.  A macro is a program that constructs another
program.  In the Forth we use the term \emph{language layer} to refer
to a collection of definitions for high level \emph{procedure} and
\emph{macro} words.

The following illustrates this principle by showing the implementation
layers in the Sheep synth, gradially moving from a low to high
abstraction.

\section{The Sheep Primitives}

The low level synth engine API\footnote{Application Programmer
  Interface. The synth API is the collection of words to interact with
  the implementation of the synth engine which itself is hidden.}
consists of the words below.
\begin{samepage}
\begin{verbatim}
engine-on  \ --   
engine-off \ --
synth      \ variable: containing the synth patch
posc0l     \ variable: low byte of OSC0 period
posc0h     \ variable: high byte ...
posc1l     \ same for OSC1
posc1h
posc2l     \ same for OSC2
posc2h
\end{verbatim}
\end{samepage}
The synth configuration (patch) is a byte with the following bit
fields.
\begin{samepage}
\begin{verbatim}
01    \ mixer: silence, xmod, reso, osc1
4     \ sync:0->1
5     \ sync:0->2
6     \ sync:1->2
7     \ osc1:noise
\end{verbatim}
\end{samepage}
The synth consists of 3 square wave oscillators that can be synced
together in different configurations, and a mixer that combines the
oscillators into a single output. The first two bits of the
configuration byte determine one of four mixing algorithms. The sync
bits determine phase reset synchronization between the 3 oscillators,
and the last can set oscillator 1 in noise generation mode.

This is our starting point. At the lowest level, the synth \emph{patch
  language} is merely a couple of configuration variables, and two
words implementing an on/off switch.  The low level implementation can
be found in \verb|synth-core.f|\footnote{This API succeeds in
  providing a simple way of producing stationary sound. In order to
  get this simplicity we give up the capability to perform arbitrary
  algorithms to make sound. By \emph{abstracting} the solution (use a
  \emph{synthesizer machine} to make some stationary sound
  parameterized \emph{only} by this configuration data), we limit the
  number of possible sounds we can make, but greatly simplify the
  expression of a useful set of abstractions that will implement
  non-stationary sounds \emph{on top} of this primitive API. The
  implementation that actually produces the stationary sound is
  hidden. It is a background task that periodically scans the
  variables to drive a couple of \emph{timers} present as hardware on
  the PIC18 chip. These timers generate \emph{interrupts} that lead to
  changes in the output state.}. On top of this primitive patch
language we will build the next level of abstraction.  In the file
\verb|synth-control.f| there is a collection of words and macros that
manipulate the configuration variables in a more high--level way.

I declined to mention a complication in the interface. Because the
synth engine updates are implemented as an \emph{interrupt service
  routine} or ISR which responds to timer interrupts, care has to be
taken that the state update is performed in a way that ensures the ISR
can see only a \emph{consistent} state\footnote{A timer interrupt can
  happen at any time, so if some part of the program is changing the 2
  state variables associated with the oscillator period, there is a
  possibility that an interrupt occurs \emph{after} writing one of the
  bytes, but \emph{before} writing the other. In other words: writing
  the 2 period values should be governed by a word that can ensure the
  operation happens \emph{atomically}, meaning the intermediate state
  is never visible to the ISR. In general, issues of assuring
  consistent used of data shared between tasks that can interrupt each
  other arbitrarily is a difficult issue. It is OK to ignore the
  details here.}. This can be done by disabling interrupts during the
write.

This is called a \emph{leaky} abstraction: the period variables are a
nice way to represent the synth config, but using them is still
cumbersome, because we need to know what rules to follow to change
their value. So what do we do? We abstract the problem! In the
\verb|synth-control.f| file there are a couple of words that perform
the correct operation for writing to period registers, without having
to worry about this interrupt business:
\begin{verbatim}
_p0 \ lo hi --    | write lo an hi period bytes for OSC0
_p1 \ lo hi --
_p2 \ lo hi --
\end{verbatim}
In the Forth the following convention is used. If a word starts with a
\emph{underscore} it will produce and/or consume \emph{double}
values\footnote{This convention is there to make a Forth mapping to
  Harvard architectures easier. PIC Microcontrollers have separate
  code and data memories that moreover do not have the same
  wordlength. Standard Forth requires code and data cells to be the
  same size. In order to keep the road open to build a Forth with a
  16--bit data cell on an 8--bit machine, in Staapl this prefix
  convention enables a simple identification of code that is 16--bit
  data cell compatible.}. For the 8-bit PIC18 Forth this means two bytes to
represent 16-bit values. For ease of use, there are also 8--bit
variants of these words:
\begin{verbatim}
p0 \ hi --    | write lo an hi period bytes for OSC0, OSC1
p1 \ hi --    | and OSC2, but compute those from a single byte
p2 \ hi --    | by multiplying it with 257
\end{verbatim}
The number 257 is chosen so a single byte can represent the entire
range of a 2--byte number. It's also simple to implemement:
multiplication by \verb|257| is the same as multiplication by
\verb|#x101| and thus a simple \verb|dup| will do the trick.

New words can now be defined to facilitate the interaction with the
words \verb|_p0|, \verb|_p1| and \verb|_p2|. One extension uses a table
internally to convert note and octave numbers to periods. See again
the \verb|synth-control.f| file for the implementation. Here I mention
just the interface:
\begin{verbatim}
  octave \ o --   | set the current octave.

  note0  \ n --   | set OSCx to the frequency 
  note1  \ n --   | of a note in the current octave,
  note2  \ n --   | counting from 0 -> C, 1 -> C#, ...
\end{verbatim}

In addition to writing, some algorithms might find a need to update
the current period instead, i.e. portamento. The following words can
be used to retreive the double words from the period registers.
\begin{verbatim}
_p0@  \ -- lo hi  | return the contents of period register for
_p1@  \ -- lo hi  | OSC0, OSC1, and OSC2
_p2@  \ -- lo hi  |
\end{verbatim}

The synth uses 4 different mixer algorithms, and they are named using
the following constants
\begin{verbatim}
macro  
: mix:silence 0 ;
: mix:xmod    1 ;
: mix:reso    2 ;
: mix:osc1    3 ;
forth    
\end{verbatim}
\verb|silence| produces a zero output, \verb|xmod| will
cross--modulate the 3 oscillators, \verb|reso| uses the oscillators to
modulate a single carrier with an envelope built from the 2 remaining
oscillators, and \verb|osc1| outputs a single oscillator for pure
noise or square wave sound. Next we build some abstractions that
modify the variable \verb|synth| and the period registers.
\begin{samepage}
\begin{verbatim}
: reso \ --
    mix:reso
    sync:0->1 bit
    sync:0->2 bit synth !

    _p0@ 
    _>> _dup _p2     \ half max reso
    _>>      _p1 ;   \ reso freq 4x
\end{verbatim}
\end{samepage}

This function creates a configuration value to store in the
\verb|synth| variable. Here the word \verb|bit \ byte bit -- byte+| is
used to set an individual bit in a byte and return the updated byte.
The constant \verb|mix:reso| representing the resonant mixer algorithm
is combined with some bits that set the sync configuration.
Additionally, we also configure the oscilattor periods. Note that
underscores represent words that operate on double size cells.  We
take the period value of OSC0, divide it by 2 using a right shift
operator and store it in OSC2's period register, then divide it by 2
again and store that result in OSC1's period register. The result is a
single word \verb|reso| that will put the synth engine into a state
that produces a sound with a clear formant, using an algorithm called
\emph{hard sync}. In \verb|synth-control.f| there are some more words
that change the synth config in a similar manner, producing some well
known synth sounds.
\begin{verbatim}
  silence \ --          | no sound
  reso    \ --          | fake resonant / formant wave
  noise   \ --          | noise
  square  \ --          | square wave    
  xmod2   \ --          | 2 OSC cross modulation
  xmod3   \ --          | 3 OSC cross modulation
  rxmod   \ --          | random cross modulation
  pwm     \ period --   | pulse width modulation
  _pwm    \ lo hi --    |   same for 16 bits
\end{verbatim}

%% \section{The Mixer Algorithms}

%% \begin{timing}[2]{1.8cm}
%% \til{1}{LHHLLHHLLHHLLHHLLHHLLHHLLHHLLHHL}
%% \til{2}{LHHHHHHHHHHLLLLLLLLLLLLLLLLLLLLL}
%% \til{3}{LHHLLHHLLHHLLLLLLLLLLLLLLLLLLLLL}
%% \end{timing}

\section{Hierarchical Time}

Next to the 3 on--chip timers used to to implement the sound
oscillators, the 4th timer in the PIC18 is used to drive a fixed rate
oscillator that provides a time base for highlevel code. It is
implemented by incrementing a 32--bit counter on each timer tick. The
counter value is stored in 4 variables \verb|tick0| to \verb|tick3|.
Now, have a look at this table which reflects binary increment from 0
to 7.
\begin{samepage}
\begin{verbatim}
000
001
010
011
100
101
110
111
\end{verbatim}
\end{samepage}
Notice that in each row, the bits change value with a period that is
the \emph{double} of that of the right neighbouring column. This
counter is a cheap source of events we can sync to, spanning a
large range of frequencies. 

This is exploited in the \verb|sync-tick| macro word, which takes a
single argument indicating which bit in the tick timer it will
synchronize on: it simply waits until that particular bit exhibits
zero to one transition. The counter is updated at a frequency of 7812
Hz = 2MHz $/$ 256, which means bit 0 has a frequency of 3906Hz. For
each bit to the left, the frequency halves. From this set of
frequencies we choose 2:
\begin{verbatim}
: wait-note    9 sync-tick ;  \   7.8 Hz
: wait-control 4 sync-tick ;  \ 244   Hz
\end{verbatim}

In the abstractions we're about the create, control rate is the rate
at which timbre modulations take place, while note rate is the
duration of a $1/16$th note at 117 BPM. These rates are rather
arbitrary, but correspond roughly to the two basic time scales that
can be used to structure sound and music. The former is useful for
creating non--stationary \emph{sounds} while the latter is useful as a
basic unit for \emph{rhythm}. This fixed relation can be a limitation,
especially for the note rate, but it severely simplifies the
implementation while still giving a useful abstration\footnote{This
  document contains a lot of examples that illustrate a common
  programming technique. Change the problem specification so the
  implementation becomes simpler. When requirements are fixed, this is
  not always possible. However, in internal representations
  (self-imposed interfaces) it usually is a good trick. For low-level
  programming, \emph{bit twiddling} hacks are a good source of such
  simplifications.}. To create non-stationary sounds, we need to
change stationary synth parameters at the proper time instances. A way
to do this is to simply put a time synchronization word in a loop:
\begin{verbatim}
: siren
    begin
        wait-note square
        wait-note noise
    again ;
\end{verbatim}
The important thing to see here is that there is is a loop with an
alternating \emph{sync} and \emph{action} part. The sync word waits
until a certain time instance, and the action part changes the state
of the sound engine to produce a different sound. When you want to
create words that have synchronization built in, where do you put the
sync part?  Before or after the action? This depends on how you want
to combine them. We'll see in the next section that the best place is
\emph{inbetween}, since that leads to easier composition of
hierarchical time scales because it avoids \emph{shared}
synchronization points.

 % shit.. there's an error still...


\section{Sound Generators}

Let's turn up the abstraction notch a bit more. In this section macros
are used to create new control flow operators. 

The Sheep is a binary output monosynth. Because mixing of sounds in
the conventional way is not possible, the only thing we can do is to
make the sound vary over time. The main problem is \emph{sequencing}
of state change events at particular time instances. Let's simplify
the problem by splitting it up into two parts, corresponding to the
two time scales identified in the previous section. We're going to
build a collection of \emph{sound generators} that implement evolution
at the control time scale and a \emph{trigger controller} that
implements changes at the note time scale.

We'll implement a sound generator a an \emph{infinite loop} that
produces an evolving non-stationary sound. At each time instance,
there is only one sound generator active. The trigger controller
activates different sound generators. The difference between the two
is that a sound generator can be \emph{started} and \emph{stopped} by
trigger controller, and that the rate of change of a sound generator
is \emph{higher} than that of a trigger controller.

This approach requires a form of multitasking, since there are two
separate threads of control. There is an infinite loop that drives the
generation of non-stationary sound, and a hierarchical control loop
that can influence the behaviour of the infinite loop. The
multitasking is implemented in the \verb|synth-soundgen.f| file and
uses functionality from \verb|pic18/task.f|. 

The interface to the sound generator mechanism consists of variable
\verb|sound| which contains a \emph{code vector} pointing to the
current loop code, and a word \verb|bang| which restarts the loop
pointed to by \verb|sound|. A code vector is a variable that contains
a pointer to a code. 

The following code uses this to create a particular sound generator
instance that produces an alternation of a square wave and a noise
burst:
\begin{verbatim}
: sync-mod
    7 sync ;
: wobble
    sound -> begin
               square 50 p0 sync-mod
               noise  sync-mod
             again
\end{verbatim}
Here we define a word \verb|sync-mod| which is used to synchronize to
a clock of about 2 Hz. This is similar to the definition of the word
\verb|wait-control| in the previous section. However, the word
\verb|sync| implements some additional behaviour compared to the word
\verb|sync-tick| used before. In addition to make the code in which it
occurs wait for the next event, it also \emph{switches} to the other
of the two tasks.

The implementation of this is fairly straightforward, but can be hard
to understand at first so I will not go into much detail\footnote{The
  multitasking works by continually switching between two tasks that
  are both waiting for the next event, and resuming operation of the
  first one that is allowed to continue. This is a simple form of
  \emph{cooperative multitasking}.  Each task has its own execution
  context, which consists of a return stack, a data stack, an
  auxiliary stack, and a separate copy of the array registers pointing
  to RAM and Flash memory. See the PIC18 Forth manual for more
  information.}. It will suffice to understand only the abstract
meaning of the \verb|sync| word. It waits for synchronization events
in the task in which it is called, and takes care of all the magic
necessary to have 2 tasks run on a single machine in the first place.

The word \verb|->| consumes a variable which contains a vector, in
this case the code vector used to store a pointer to the current sound
generator loop code, and sets it to point to the code immediately
following the arrow. The word \verb|wobble| in which \verb|->| occurs
is then aborted immideately, which means the code behind it will not
run as part of the execution of \verb|wobble|, but it is only used to
define the meaning of the variable passed to \verb|->|.  The word
\verb|wobble| merely sets the \emph{value} of the variable
\verb|sound|; it only changes the state of the sound generator. The
next time \verb|bang| is executed, this state is collected from the
variable \verb|sound| and the loop wil start running. Note that the
generator loop and the trigger controller are tied: the loop only runs
as long as a trigger controller is active, because it is activated
during the time the controller word is waiting for a next event.  This
is a simple trigger controller:
\begin{samepage}
\begin{verbatim}
: note-sync
    9 sync ;
: notes \ n -- 
    wobble               \ make the wobble sound current
    bang                 \ restart current sound
    for note-sync next   \ wait for n notes
    silence ;            \ turn off the sound
\end{verbatim}
\end{samepage}
Running \verb|10 notes| will produce a wobbly sound for a duration of
10 notes. Note that a sound generator acts as a \emph{virtual sample
  player}. From the viewpoint of the controller, the sound is simply
turned on and off. From the viewpoint of the sound generator, it will
loop forever. This abstraction separates concerns at the the
note--level time scale from concerns at the time scale at which sound
modulations happens.


% \section{Transients}

\section{Pattern Programs}

Let's recall the tower of abstractions we've used uptil now. The
stationary sound synth is abstracted through a number of configuration
variables. On top of this we've built an abstraction that allows the
hierarchical sequencing of note rate events that trigger sounds
implemented by loops that modulate the stationary sound synth at
modulation rate.

Instead of writing words which perform an entire piece in the way
explained above, where all the parts have to be expressed
\emph{statically} beforehand, it might be interesting to add the
ability to change the behaviour of words at run time. This is called
\emph{dynamic binding}. Note that the \verb|->| mechanism introduced
above is already an example of this.

In this section we will explore the use of \emph{byte codes}. A byte
code is a number which represents a certain behaviour. The idea is
that the relationship between a number and its associated behaviour
(word) is fixed, but the numbers themselves can be stored in variables
to implement a \emph{state machine}. A state machine is a machine
which has a variable behaviour that depends on its current state.

In the Forth this is implememented using the \verb|route| macro word.
This word can be used to implement jump tables that associate a number
to an array of words implementing behaviour\footnote{This is
  implemented by skipping a number of machine instructions. It so
  happens that a word followed by a semicolon is exactly one machine
  instruction wide. It is a jump to the code that implements that
  word.  An ``upside down'' semicolon behaves the same, but indicates
  to the compiler that the following code is still accessible.}. The
words in a jump table are separated by \verb|.,| while the last word
is terminated using \verb|;|
\begin{verbatim}
: step \ n -- 
    route
      left ., right ., up ., down ;
\end{verbatim}
The word \verb|step| accepts a number which it maps to behaviour. In
this case 0 is mapped to \verb|left|, 1 to \verb|right|, \ldots The
end result is that we can write code that does something based on a
\verb|variable| that can be changed at run time. For example
\begin{verbatim}
variable direction
: go  direction @ step ;
\end{verbatim}
Now \verb|go| will perform one of 4 behaviours depending on the value
of the variable \verb|direction|.

We now create a language for expressing rhythmic patterns on top of
the abstractions created thusfar. Suppose the word \verb|drum|
executes some synth reconfiguration event that changes the
\verb|sound| vector. A way to build a pattern sequencer is to create a
map from a variable \verb|time| to behaviour. Note that an upside down
semicolon all by itself is just a RETURN instruction and thus does
nothing.
\begin{samepage}
\begin{verbatim}
variable time
: a-sequence time @
    route
        drum ., drum ., drum ., drum .,
        drum .,      .,      .,      .,
        drum ., drum .,      ., drum .,
        drum .,      ., drum .,      ;
\end{verbatim}
\end{samepage}
This implements a state machine \verb|a-sequence|. Its behaviour
depends on the state variable \verb|time|.

For convenience, let's limit the size of the patterns to 16. One thing
I neglected to mention is that you can easily jump off the end of a
jump table if the number is too big. Solve this by performing an
\verb|#x0F and| operation after fetching the byte. In that case one
never has to worry about the value being out of range, and
incrementing time is just \verb|time 1+!|, resulting in a looping
pattern.

Now, let's introduce some macros. Note that a macro is just an
abbreviation for a sequence of words. This makes it similar to a
procedure word, however macros can implement behaviour that simple
procedure words cannot. In general, you need to use macros to build
abstractions on top of other macros. So to compose \emph{control
  words}, which are implemented as macros, one needs to use
macros. Control words change the normal sequential order of
execution. In the following we will build abstractions around the
macros \verb|.,| and \verb|route|\footnote{It might help to have a
  look at the PIC18 Forth paper and see how basic control flow
  operations are implemented. The fact that these constructs are
  programmable is one of the most powerful elements of Forth in
  general.}.
\begin{samepage}
\begin{verbatim}
variable  time
macro
: o       drum ., ;
: .       ., ;
: pattern time @ #x0F and route ;
forth
\end{verbatim}
\end{samepage}
With these macro definitions, each occurence of \verb|o| will be
replaced with \verb|drum .,| and each occurence of \verb|.| with
\verb|.,| and similarly each occurance of \verb|pattern| will be
replaced \verb|time @ #x0F and route|. This means the word
\verb|a-sequence| is equivalently defined as:
\begin{samepage}
\begin{verbatim}
: a-sequence 
    pattern
      \ 1  2  3  4  5  6  7  8  9 10 11 12 13 14 15 16 
        o  o  o  o  o  .  .  .  o  o  .  o  o  .  o  . ;
\end{verbatim}
\end{samepage}
That's an ASCII art GUI for creating sequencer patterns! Note that
what we did is to simply identify reusable structure in the code and
name it appropriately. Wheter this name is then implemented as a macro
or a procedure depends on what functionality it abstracts, but the
mechanism is largely the same.

Now, to complete the story, let's implement the word \verb|drum| so we
can reuse the word \verb|a-pattern| for different sounds.  We'll use
the same vector mechanism that was used to implement \verb|sound| in a
previous secion. What I didn't mention before is that vectors can be
executed using the \verb|invoke| word. The complete implementation of
\verb|drum| and the associated macro \verb|o| is:
\begin{verbatim}
vector drum
: do-drum  drum invoke ;   
macro
: o        do-drum ., ;  
forth

\ some words that change the drum sound
: snare    drum -> init-drum-sound ;
: hihat    drum -> init-hihat-sound ;

\end{verbatim}
Excuting the \verb|do-drum| word sets the variable \verb|drum| to
point to the code sequence \verb|init-drum-sound|.  The code sequence
\verb|snare a-sequence| would execute the time--variant word
\verb|a-sequence| defined earlier. If the variable \verb|time|
contains a state number that corresponds to a \verb|o| in the pattern,
the word \verb|init-drum-sound| will be executed. Alternatively, if
the pattern contains a \verb|.| word at that place, this word won't
execute.

What we just did is to parameterize the behaviour of \verb|a-sequence|
using the word \verb|snare|. Similarly, the word \verb|hihat| can be
used to do the same.

\section{Conclusion}

The examples in this tutorial illustrate the construction of a DSL for
building a sound synthesizer from almost nothing. Compared to ordinary
high level programming, where the abstraction level one starts from is
already quite high, this is simply barbaric.

However, it is my opinion that the level of abstraction at which one
starts doesn't really make much of a difference. The trick to writing
good software is to learn how to use the primitives and the
composition mechanisms effectively. This is not central to Forth, but
Forth allows to do this in a very elegant way.

I hope to show with this tutorial that all you really need to build a
powerful DSL from almost nothing is a composition mechanism that is
powerful. For concatenative languages, this composition mechanism is
string substitution: a word can refer to a sequence of words. This can
then be implemented in two ways. Procedure words will be substituted
at program run time using the chip's procedure call abstraction, while
macro words are substituted at program compile time.



\end{document}
